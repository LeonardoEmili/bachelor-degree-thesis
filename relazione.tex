% !TeX encoding = UTF-8
% !TeX program = pdflatex
% !TeX spellcheck = it_IT

\documentclass[binding=0.6cm,Lau,noexaminfo]{sapthesis}

\usepackage{microtype}
\usepackage[italian]{babel}
\usepackage[utf8]{inputenc}
\usepackage[hidelinks]{hyperref}
\usepackage{setspace}
\onehalfspacing

\hypersetup{pdftitle={Progettazione di un dispositivo IoT basato su wake up radio},pdfauthor={Leonardo Emili}}

\title{Progettazione di un dispositivo IoT basato su wake up radio}
\author{Leonardo Emili}
\IDnumber{1802989}
\course{Informatica L-31}
\courseorganizer{Facoltà di Ingegneria dell'informazione, informatica e statistica}
\AcademicYear{2019/2020}
\copyyear{2020}
\advisor{Prof. Chiara Petrioli}
\tutor{Dr. Georgia Koutsandria}
\tutorcoord{Prof. Chiara Petrioli}
\tutorcoord{Dr. Georgia Koutsandria}
\authoremail{emili.1802989@studenti.uniroma1.it}

%\schoolname{Sapienza, Università di Roma}
%\schooladdress{Piazzale Aldo Moro 5, 00185 Roma}
%\schoolprincipal{Dr. Alessandro Mei, PhD}

\begin{document}
\large

\frontmatter
\maketitle
%\dedication{Dedicato a\\ Donald Knuth}

\tableofcontents

\mainmatter
\chapter{Introduzione}

%Sviluppo dell'Iot nel mondo: quando, dove perchè e come.
%WSN reti di nodi sensori con applicazioni in, permettono di, hanno queste sfide (energia, latenza, disponibilità)
%Nel mio percorso mi sono concentrato su green WSN: studio di protocolli e delle metodologie impiegate, implementazione di GreenWup un protocollo fatto così, e variazioni.

Nell'ultimo decennio abbiamo assistito ad un portentoso sviluppo del settore dell'Internet of Things che ha contribuito alla diffusione
di un enorme numero di dispositivi wireless. Il numero di questi dispositivi ha registrato crescite costanti negli anni e loro applicazioni sono pressoché infinite.
Essi trovano impiego in ambito domestico dove realizzano l'automatizzazione dei compiti quotidiani, in quello sanitario in cui monitorano lo stato di salute
dei pazienti e in ambito sottomarino dove gli obbiettivi spaziano da quello di realizzare campionamenti di dati sino a creare vaste reti di comunicazione sottomarine.
L'immissione di questo enorme contingente di dati nella rete ha reso possibili nuove interpretazioni e lo sviluppo di soluzioni che puntano a
migliorare la qualità della vita delle persone, anche in quei settori noti esser di difficile comprensione. Ad esempio, l'impiego di sensori IoT in ambito di
monitoraggio del crosta terrestre ha reso disponibili informazioni fino a prima sconosciute sulla presenza di terremoti e tsunami, migliorando radicalmente
la nostra percezione degli eventi sismici nel mondo.\\

Con lo sviluppo massivo dell'IoT nuove sfide sono emerse a minare la solidità degli approcci usati. Attualmente le soluzioni impiegate nello sviluppo
dei dispositivi wireless puntano a realizzare comunicazioni con basse latenze e che siano altamente efficienti in termini energetici. In ambito di ricerca,
lo stato corrente dell'arte punta a realizzare tecnologie hardware e software che implementino questo binomio. Un particolare
settore dell'IoT si occupa di realizzare vaste reti di nodi sensori interconnessi, note come \emph{Wireless Sensor Networks},
che realizzano una comunicazione wireless con consumi minimi al fine di prolungare i tempi di servizio dell'intera rete. In questo caso,
le principali problematiche sono chiaramente settoriali e sono identificate dallo specifico campo di applicazione: come la richiesta di tecnologie
altamente performanti per i monitoraggio clinici oppure di altre che garantiscano elevate lifetime nel caso di sensori ambientali. Tuttavia esistono
problematiche comuni a queste tipologie di reti, ad esempio spesso si richiede che l'intera rete sia connessa e che tutti i nodi siano,
anche parzialmente, a conoscenza della topologia della rete. Nel caso, si richiede l'impiego di protocolli che ne prevedano il
costante aggiornamento e favoriscano una comunicazione efficiente. Inoltre, in queste reti la principale fonte di consumo energetico risiede
nella comunicazione e nell'attesa che la precede, è quindi di vitale importanza prevedere che i nodi della rete siano
oggetto di scaricamento delle batterie che li alimentano. Infatti, la morte prematura di un nodo può spesso avere conseguenze più grandi della semplice
disconnessione dello stesso, poichè frequentemente si tratta di reti \emph{multi-hop} che realizzano la comunicazione passando attraverso nodi intermedi, si può
verificare perfino la disconnessione dell'intera rete.\\


Nel capitolo 2 ...\\

Nel capitolo 3 ...\\

Nel capitolo 4 ...\\


\chapter{Scenario di riferimento}

Sino a qualche anno fa, la principale tendenza nello sviluppo di soluzioni ai problemi sopra menzionati consisteva nell'introduzione di cicli di attività
dei nodi, il cosiddetto \emph{duty cycling} definito come la frazione di tempo in cui un nodo è attivo, il cui obbettivo era quello di minimizzare i momenti
in cui i nodi sensori sono accessi senza alcun processamento di dati attivo. Seppur questo approccio permetta di prolungare la durata della vita della rete,
incrementa in maniera considerevole i ritardi nelle comunicazioni, dal momento che queste possono avvenire solamente all'interno delle finestre di attività
delle coppie dei nodi coinvolti. Inoltre l'utilizzo del duty cycling non permette di risolvere i problemi sopra citati, in quanto le comunicazioni
possono ancora avvenire nei momenti di inattivà e dualmente esistono ancora i periodi di tempo in cui i nodi consumano energia pur non ricevendo o inviando dati.
In questo panorama vengono introdotte le \emph{wake up radios}, in grado di abbattere i consumi energetici mantenendo bassi i ritardi nelle comunicazioni.
La principale differenza rispetto all'approccio tradizionale consiste nell'introdurre un'antenna secondaria per i soli messaggi di wake-up che è costantemente
attiva. La nuova antenna è progettata per avere di dimensioni e consumi energetici nettamente inferiori rispetto alla radio principale e, al momento della ricezione
di un messaggio di wake-up, è possibile attivarla e ricevere il messaggio dati. Di fatto, questa tecnologia implementa uno schema di comunicazione on demand:
un nodo può inviare un pacchetto dati ad un nodo dormiente semplicemente  posticipandone l'invio a quello di una sequenza di wake-up che, svegliando il ricevente,
lo abilita alla ricezione del pacchetto dati. In questo modo i nodi possono rimanere attivi per il solo tempo minimo necessario a svolgere l'attività richiesta
e lasciare spenta la radio principale nel tempo rimanente, essendo eventualmente svegliati da una sequenza di wake-up quando richiesto.
Dal punto vista energetico questo approccio è altamente efficiente e non richiede alcuna forma di sincronizzazione che è generalmente non desiderata in quanto
implicitamente introduce overhead dovuto alla gestione di timer aggiuntivi.\\

Più recentemente la tendenza è quella di equipaggiare i nodi sensori, dotati di wake up radios, con moduli per l'\emph{energy harvesting} come ulteriore
supporto energetico alla loro durata di vita. In particolare, i nodi sono in grado utilizzare l'energia fornita dalle pile elettriche di cui sono
forniti e derivarne di nuova mediante turbine eoliche, pannelli solari e generatori termoelettrici di cui sono equipaggiati. La ricerca dimostrato che l'impiego
di nodi dotati di energy harvesting \cite{energy-harvesting-paper} e che sono abilitati alle wake up radios \cite{wake-up-radios-paper} ha portato notevoli
incrementi nelle performance delle WSN. Inoltre, l'innovativa tecnologia delle wake up radios può essere combinata con il cosiddetto \emph{semantic addressing}.
L'idea è di limitare ad un sottoinsieme di nodi la ricezione di una sequenza di wake up, evitando quindi che questa sia recepita da tutti i nodi presenti all'interno
del range di comunicazione. Il principio fondante in questo caso è di scegliere
opportunamente una sequenza di wake up da assegnare ad un certo numero di nodi della rete e di svegliare, al momento della ricezione del messaggio, tutti e soli i
nodi desiderati. In questo scenario, i consumi energetici possono essere ulteriormente ridotti poichè virtualmente si azzerano il numero dei nodi che vengono
riattivati a seguito di comunicazioni che non sono a loro destinate.\\

Questo lavoro si concentra sullo studio dei cosiddetti \emph{green protocols}, ovverosia di quei protocolli che si basano sulle tecnologie sopra esposte
e che implementano la comunicazione tra nodi minimizzandone gli sprechi energetici e massimizzando la durata della vita delle reti. Nella letteratura scientifica
sono presenti molti protocolli di questo tipo (es. CTP-WUR, GREENROUTES, WHARP) ed ognuno punta a risolvere problemati sollevate altri, migliorandone le performance
rispetto a diverse metriche. Nel seguito verrà analizzato in dettaglio il comportamento del protocollo GREEN-WUP che utilizza energy harvesting e
wake up radios per realizzare un'opportuna scelta dei nodi intermedei necessari alla ricezione dei dati da parte del sink.\\

\chapter{Il protocollo GREEN-WUP}
\chapter{Soluzioni proposte}
\chapter{Valutazione delle prestazioni}
\chapter{Conclusioni}

\backmatter
\cleardoublepage
\phantomsection % Give this command only if hyperref is loaded
\addcontentsline{toc}{chapter}{\bibname}

\begin{thebibliography}{9}

    \bibitem{energy-harvesting-paper}
    Stefano Basagni, Georgia Koutsandria, Chiara Petrioli.
    \textit{A Comparative Performance Evaluation of Wake-up Radio-based Data Forwarding for Green Wireless Networks}.

    \bibitem{wake-up-radios-paper}
    Stefano Basagni, Valerio Di Valerio, Georgia Koutsandria, Chiara Petrioli.
    \textit{Wake-up Radio-enabled Routing for Green Wireless Sensor Networks}
\end{thebibliography}

\end{document}