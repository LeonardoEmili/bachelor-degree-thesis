% !TeX encoding = UTF-8
% !TeX program = pdflatex
% !TeX spellcheck = it_IT

\documentclass[binding=0.6cm,TFA]{sapthesis}

\usepackage{microtype}
\usepackage[italian]{babel}
\usepackage[utf8]{inputenc}
\usepackage[hidelinks]{hyperref}
\hypersetup{pdftitle={Il protocollo GreenWup, approcci e performance},pdfauthor={Leonardo Emili}}

\title{Il protocollo GreenWup, approcci e performance}
\author{Leonardo Emili}
\IDnumber{1802989}
\course{Informatica L-31}
\courseorganizer{Facoltà di Ingegneria dell'informazione, informatica e statistica}
\AcademicYear{2019/2020}
\copyyear{2020}
\advisor{Dr. Chiara Petrioli, PhD}
\tutor{Georgia Koutsandria, PhD}
\tutorcoord{Dr. Chiara Petrioli, PhD}
\tutorcoord{Georgia Koutsandria, PhD}
\authoremail{emili.1802989@studenti.uniroma1.it}

%\schoolname{Sapienza, Università di Roma}
%\schooladdress{Piazzale Aldo Moro 5, 00185 Roma}
%\schoolprincipal{Dr. Alessandro Mei, PhD}

\begin{document}

\frontmatter
\maketitle
%\dedication{Dedicato a\\ Donald Knuth}

\tableofcontents

\mainmatter
\chapter{Introduzione}

%Sviluppo dell'Iot nel mondo: quando, dove perchè e come.
%WSN reti di nodi sensori con applicazioni in, permettono di, hanno queste sfide (energia, latenza, disponibilità)
%Nel mio percorso mi sono concentrato su green WSN: studio di protocolli e delle metodologie impiegate, implementazione di GreenWup un protocollo fatto così, e variazioni.

Nell'ultimo decennio si è assistito ad un portentoso sviluppo del settore dell'Internet of Things che ha contribuito alla diffusione
di un enorme numero di dispositivi wireless. Il numero di questi dispositivi ha registrato crescite costanti negli anni e loro applicazioni sono ormai infinite.
Essi trovano impiego in ambito domestico dove realizzano l'automatizzazione dei compiti quotidiani, in quello sanitario in cui monitorano lo stato di salute
dei pazienti e in ambito sottomarino dove l'obbiettivo spazia dal realizzare campionamenti di dati sino a creare vaste reti di comunicazione sottomarine.
L'immissione di questo enorme contingente di dati nella rete ha reso possibili nuove interpretazioni e lo sviluppo di soluzioni che puntano a 
migliorare la qualità della vita delle persone, anche in quei settori noti esser di difficile comprensione. Ad esempio, l'impiego di sensori IoT in ambito di
monitoraggio del crosta terrestre ha reso disponibili informazioni prima sconosciute sulla presenza di terremoti e tsunami, migliorando radicalmente
la nostra percezione degli eventi sismici nel mondo.\\

Con lo sviluppo massivo dell'IoT nuove sfide sono emerse a minare la solidità degli approcci usati. Attualmente le soluzioni impiegate nello sviluppo
dei dispositivi wireless puntano a realizzare comunicazioni a bassa latenza ed altamente efficienti in termini energetici. In ambito di ricerca,
lo stato corrente dell'arte punta a realizzare tecnologie hardware e software che implementino questo binomio. Ad esempio, un particolare
settore dell'IoT si occupa di realizzare vaste reti di nodi sensori interconnessi, note come \emph{Wireless Sensor Networks}, 
che realizzano una comunicazione wireless con consumi minimi al fine di prolungare i tempi di servizio dell'intera rete. In questo caso,
le principali problematiche sono chiaramente settoriali e sono identificate dallo specifico campo di applicazione: la richiesta di tecnologie
altamente performanti per i monitoraggio clinici ed di altre che garantiscano durate elevate nel caso di sensori ambientali. Tuttavia esistono
problematiche comuni a queste tipologie di reti, ad esempio spesso si richiede che l'intera rete sia connessa e che tutti i nodi siano,
anche parzialmente, a conoscenza della topologia della rete. Nel caso, si rende l'impiego di protocolli che ne prevedano il
costante aggiornamento e favoriscano una comunicazione efficiente. Inoltre, in queste reti la principale fonte di consumo energetico risiede
nella comunicazione e nell'attesa che la precede, è quindi di vitale importanza prevedere che i nodi che fanno parte della rete siano
oggetto di scaricamento delle batterie che li alimentano. Infatti, la morte prematura di un nodo può spesso avere conseguenze più grandi della semplice
disconnessione dello stesso, poichè frequentemente si tratta di reti \emph{multi-hop} che realizzano la comunicazione passando attraverso nodi intermedi, e si può
verificare una disconnessione dell'intera rete.\\

Sino a qualche anno fa, la principale tendenza nello sviluppo di soluzioni ai problemi sopra menzionati consisteva nell'introduzione di cicli di attività
dei nodi, il cosiddetto \emph{duty cycling} definito come la frazione di tempo in cui un nodo è attivo, il cui obbettivo era quello di minimizzare i momenti
in cui i nodi sensori sono accessi senza alcun processamento di dati attivo. Seppur questo approccio permetta di prolungare la vita della rete,
incrementa in maniera considerevole i ritardi nelle comunicazioni, dal momento che queste possono avvenire solamente all'interno delle finestre di attività
delle coppie dei nodi coinvolti. Generalmente la richiesta di sincronizzazione delle parti è una proprietà scarsamente desiderata, in quanto implicitamente
introduce overhead dovuto alla gestione di timer. Inoltre l'utilizzo del duty cycling non permette di risolvere i problemi citati, in quanto le comunicazioni
possono ancora avvenire nei momenti di inattivà e dualmente esistono ancora i periodi di tempo in cui i nodi consumano energia pur non ricevendo o inviando dati.
In questo panorama vengono introdotte le \emph{wake up radios}, che permettono di abbattere i consumi energetici mantenendo bassi i ritardi nelle comunicazioni.
Di fatto, queste implementano uno schema di comunicazione on demand: un nodo che deve inviare un pacchetto dati ad un nodo dormiente può farlo posticipandone
l'invio dopo quello di un pacchetto di wake-up che sveglia il ricevente, abilitandolo alla ricezione del pacchetto dati. In questo modo i nodi possono
rimanere attivi solo per il tempo minimo richiesto all'attività che devono svolgere ed essere eventualmente svegliati da una sequenza di wake-up nel caso
in cui un nodo gli richieda di ricevere dati. Dal punto vista energetico questo approccio è altamente efficiente e non richiede alcuna forma di
sincronizzazione. Si rende quindi possibile attraverso l'impiego delle wake up radios l'implementazione di reti di nodi sensori ad alta efficienza energetica ...

\backmatter
\cleardoublepage
\phantomsection % Give this command only if hyperref is loaded
\addcontentsline{toc}{chapter}{\bibname}
% Here put the code for the bibliography. You can use BibTeX or
% the BibLaTeX package or the simple environment thebibliography.

\end{document}