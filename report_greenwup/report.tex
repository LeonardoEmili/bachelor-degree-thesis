\documentclass{report}

\usepackage[italian]{babel}
\usepackage[utf8]{inputenc}
\usepackage[hidelinks]{hyperref}
\usepackage{graphicx}
\usepackage[font=small,labelfont=bf]{caption}
\usepackage{setspace}
\onehalfspacing

\graphicspath{ {../assets/} }

\title{Report sul protocollo GREEN-WUP}
\author{Leonardo Emili}

\begin{document}
\maketitle
\tableofcontents

\chapter{GREEN-WUP}
\section{Il protocollo GREEN-WUP}

Il protocollo GREEN-WUP si inserisce nel contesto dei protocolli delle cosiddette \emph{green wireless sensor network} e pone tra i suoi principali
obbiettivi l'efficienza energetica dell'intera rete. Esso impiega le tecnologie di wake up radio, energy harvesting e semantic addressing. Inoltre, è 
di tipo \emph{converge casting} ed è basato sull'assegnazione di hop count a ciascun nodo per poter distribuire i pacchetti dati
all'interno della rete.\\

Esso si articola in due fasi principali: in primo luogo avviene la fase di \emph{interest dissemination} dove si definisce la
topologia della rete da rispettare affinchè i nodi realizzino un corretto flusso di scambio dati; successivamente si assume che gli indirizzi di
wake up siano stati assegnati e si procede con lo scambio di dati che è governato da sequenze di wake up che vengono utilizzate per
risvegliare i nodi della rete.\\

Inizialmente il sink node avvia la fase di  inviando
un primo \emph{command packet} che viene successivamente distribuito ai nodi sensori utilizzando l'algoritmo di \emph{flooding}.
L'obbiettivo di questo processo è di assegnare a ciascun nodo della rete un valore di hop count secondo un algoritmo iterativo: questo
avrà valore 0 nel solo caso del sink node, altrimenti assume un valore $h$ se $h-1$ è il valore di hop count del nodo precedente. Al termine
di questa fase ciascun nodo sarà fornito un indirizzo di wake up $w=w_{h}w_{e}$ della lunghezza di 8 bit, con $w_{h}$ definito dalla
codifica in 4 bit di $h$ e $w_{e}$ che rappresenta la classe energetica del nodo, codificata nei restanti 4 bit. In particolare, ciascun nodo considera
l'energia disponibile come quella rimamente nelle batterie assieme a quella derivata da sorgenti esterne. In definitiva, la codifica del suffisso $w_{e}$
viene calcolata a partire dalla discretizzazione in $k$ classi della disponibilità energetica di un nodo, dove $k$ rappresenta il numero delle classi
disponibili.\\

Al termine della fase di interest dissemination ciascun nodo è in grado di calcolare il proprio indirizzo di wake up, che sarà periodicamente aggiornato in base alla loro
disponibilità energetica corrente. Questa idea realizza il principio del semantic addressing poiché in questo scenario
è possibile far riferimento ad un sottoinsieme di nodi della rete a partire dai loro valori di hop count e da quello della classe energetica. \\

%image source: https://www.researchgate.net/figure/Topology-of-wireless-sensor-network-and-hop-count-of-sensors_fig7_285956270
\begin{figure}
    \begin{center}
        \includegraphics[scale=1.7]{hop-count-algorithm.png}
        \caption{Topologia della rete a seguito dell'assegnazione degli hop count}
    \end{center}
\end{figure}

In definitiva, se un nodo deve inviare un pacchetto dati può farlo seguendo degli step fondamentali: una prima fase di comunicazione con i nodi elegibili
alla ritrasmissione del pacchetto; una seconda fase di selezione del nodo che si farà carico della richiesta; infine la fase di trasmissione del pacchetto,
a seguito della quale verrà inviata una conferma a certificare l'avvenuta ricezione dello stesso.\\

In particolare,
consideriamo un nodo $a$ non a diretto collegamento col sink node, dunque si ha un valore di hop count $l>1$. Al momento della richiesta di trasmissione dati,
$a$ prepara una sequenza di wake up $w$ con semantic addressing, scegliendo inizialmente la massima classe energetica $k$, e verifica facendo
\emph{carrier sensing} che il canale di comunicazione sia libero. Se il canale è libero allora procede inviando la sequenza di wake up che sveglierà tutti
e soli i nodi con classe energetica massima. Esso attende un contingente di tempo necessario affinchè i nodi destinatari, siano questi $B_1, B_2, \ldots B_n$ 
possano accendere le antenne principali (RX) ed invia in broadcast un pacchetto \emph{Request To Send} (RTS). Al momento della ricezione 
i nodi $B_1, B_2, \ldots B_n$ spengono la radio principale (SLEEP). Successivamente inviano al nodo $a$ una sequenza di wake up seguita da un pacchetto
\emph{Clear To Send} (CTS), utilizzando come indirizzo quello contenuto all'interno del pacchetto RTS precedentemente ricevuto. Per evitare collisioni
durante l'invio del pacchetto CTS i nodi ne rimandano la trasmissione utilizzando un tempo di \emph{jitter} randomico.
Al momento della ricezione del primo CTS, $a$ seleziona il nodo $B_i$ che lo ha inviato per la ricezione del pacchetto dati.
Quindi $a$ verifica che il canale di comunicazione sia libero ed in caso positivo invia a $B_i$ una sequenza di wake up
seguita dal pacchetto dati. Dunque, $B_i$ ricevuto il pacchetto dati invia un \emph{acknowledgement packet} (ACK) ad $a$, va a dormire e prosegue la
ritrasmissione del pacchetto ricevuto sino ad arrivare al sink node. Infine, $a$ va a dormire in attesa di svolgere una nuova operazione.\\

Il design del protocollo prevede che ciascun nodo consideri dapprima i soli nodi con classe energetica più alta. Se, a seguito di un numero massimo di tentativi,
questo non riesce a mettersi in contatto con nessuno di loro allora si procede considerando la classe energetica immediamente inferiore e si ripete
l'intera procedura. Inoltre, nel caso in cui il nodo sorgente è a diretto contatto con il sink node ($l=1$), il protocollo prevede che
lo scambio dei pacchetti DATA e del relativo ACK avvenga senza avviare la procedura di scambio di RTS e CTS, in quanto superflua la
selezione di un nodo intermedio.\\

\section{Le problematiche emerse}

GREEN-WUP impiega jitter puramente randomici al momento dell'invio dei pacchetti CTS per tentare di evitare che avvengano collisioni durante
la trasmissione. In questo modo, i nodi inviano un pacchetto CTS il cui tempo di ricezione dipende unicamente dal valore di jitter
considerato e dalla distanza dei singoli dal nodo ricevente, in quanto viene considerato il tempo di trasmissione un'ulteriore ritardo nella
trasmissione pacchetto. La scelta \emph{greedy} secondo cui un nodo trasmette un pacchetto ad un certo tempo $t+jitter$ non aggiunge alcuna
informazione sulla stato dello stesso. Si tratta di una scelta per alcuni aspetti ragionevole, in quanto non aggiunge overhead alla trasmissione
del pacchetto, ma che non garantisce che il percorso scelto nella rete sia il migliore in termini di efficienza e costo energetico. Un nodo
potrebbe essere selezionato come nodo intermedio seppur trovandosi in uno stato di scarsa disponibilità energetica. Considerando inoltre che 
ciascun nodo è dotato di un buffer di ricezione con capacità limitata, è possibile che un nodo sia selezionato come nodo intermedio pur non 
potendo bufferizzare nuovi pacchetti a seguito di un riempiemento della coda (\emph{buffer overflow}).\\

Inoltre, durante lo scambio di RTS e CTS viene scelto un nodo che opererà da intermediario per la ricezione del pacchetto destinato
al sink node. I nodi sono considerati sulla base della loro posizione rispetto al sink node (HOP COUNT) e alla loro disponibilità energetica corrente.
Tuttavia questa procedura non garantisce che tutti i nodi svegliati siano abilitati alla ricezione di nuovi pacchetti, poichè, in maniera simile a
quanto sopra descritto, un nodo può essere nella condizione di non poter ricevere nuovi pacchetti, verificandosi quindi sprechi di energia ed
ulteriori ritrasmissioni per far recapitare il pacchetto. In generale, l'approccio di GREEN-WUP non garantisce l'efficienza energetica della rete,
nè tantomeno permette di ottenere latenze ottimali. Si osservi come un nodo che è impegnato nel processamento di un pacchetto possa essere selezionato
come nodo intermediario rispetto ad un altro che si trova alla stessa distanza dal sink node (stesso hop count) e che ha coda di ricezione vuota, se questo
ha, ad esempio, classe energetica immediamente inferiore al nodo in questione. \\

Osservando in dettaglio il comportamento dei nodi durante la fase di selezione del nodo intermediario si può notare il seguente dettaglio.
Al momento della ricezione di una sequenza di wake up con semantic addressing da parte di un nodo, questo attiverà la radio principale e si metterà in
ascolto di nuovi dati. Questo accade poichè la risposta al trigger della sequenza di wake up genera un evento in ciascun nodo che lo porta inevitabilmente
ad attivare la radio principale. Se tuttavia i nodi candidati a svegliarsi in seguito alla ricezione del messaggio di wake up sono numerosi si può avere uno
spreco energetico considerevole. Un'analisi del comportamento di GREEN-WUP ci permette di osservare come i nodi riceventi non hanno possibilità di scegliere l'azione
intrapresa in seguito alla ricezione di un messaggio di wake up. Essi semplicemente reagiscono al messaggio ricevuto e si preparano alla ricezione di un
eventuale pacchetto dati. Analogamente alla prima problematica emersa, la scelta da parte di un nodo di svegliarsi è effettuata senza considerare lo stato dello
stesso e può portare a consumi energetici superflui.

\section{Soluzioni proposte e risultati}

Le proposte avanzate sono state testate attraverso il framework \emph{GreenCastalia} che permette di eseguire simulazioni di WSN che utilizzano le tecniche
di \emph{energy harvesting} e \emph{wake up radios}. I risultati e i dati di seguito forniti sono considerati su un numero minimo di simulazioni
necessario a fornire veridicità alle considerazioni affermate, vedere più avanti la formula utilizzata.\\

In questo lavoro si propone una variante del protocollo di base di GREEN-WUP che adotta alcune modifiche in risposta alle problematiche emerse nel capitolo
precedente. Nel seguito, per ciascuna problematica si descrivono le modifiche adottate, le implicazioni e i risultati che questi cambiamenti hanno comportato.\\

La variante proposta risolve il precedente problema della mancanza di stato nella formulazione del valore di jitter considerando lo stato energetico
corrente del nodo e lo stato della sua coda di ricezione. In particolare, un nodo sarà tanto più prono a ricevere e reinstradare un pacchetto quanto
più è alta è la sua disponibilità energetica e di ricezione di nuovi pacchetti. Un nodo $j$ risponde ad un pacchetto RTS inviando un pacchetto CTS
ad un tempo $t+\delta$, con $\delta$ così definito:

$$ \delta= (1-e_{j}) \cdot k \cdot \delta_{M} + b_{j} \cdot l \cdot \delta_{M} + \delta_{r} \quad ,$$

con $e_{j}$ la frazione di energia residua del nodo corrente, $b_{j}$ lo stato di occupazione del buffer, $\delta_{M}$ il massimo delay consentito,
$\delta_{r} < \delta_{M}$ un valore randomico ed infine $k \geq 0$ e $l \geq 0$, con $k+l=1$, che regolano il contributo di ciascun elemento.\\

La presenza di un offset randomico non influisce sulla correttezza dell'approccio impiegato. Senza perdita di generalità possiamo considerarlo come una
costante dal momento che il suo unico scopo è quello di introdurre un evento aleatorio al fine di evitare le collisioni derivate dall'invio di pacchetti CTS
nello stesso momento da parte di nodi nello stesso stato (pari energia residua e stato del buffer). Così facendo il jitter scelto cresce al crescere
delle due componenti considerate. Si noti come nell'implementazione si è scelto di dare pari peso ad entrambe le componenti ($k=1/2=l$) per permettere il
tradeoff con bassi valori per la latenza e contemporaneamente diminuire i consumi energetici della rete.\\

Un altro obbiettivo della variante del protocollo di base è quello di limitare i consumi energetici della rete dovuti a risvegli non necessari dei nodi
della rete. Quindi, l'idea del semantic addressing sulla quale si basa il protocollo GREEN-WUP viene espansa introducendo lo stato del buffer di ricezione dei nodi
destinatari. L'indirizzo di wake up che viene inviato per svegliare i nodi che prima consisteva nel valore di hop count e nella classe energetica del nodo destinario
viene modificato nel formato: $1BBHHHEE$. Con $B$ che rappresenta la classe dello stato del buffer di ricezione del nodo, $H$ che rappresenta l'hop count ed
$E$ che rappresenta la classe energetica. Si noti come la presenza del prefisso 1 serva per il rilevamento dell'energia del nodo secondo la \emph{OOK modulation}.
L'aggiunta dei bit per lo stato del buffer permettono di selezionare dapprima i nodi con classe energetica superiore che siano disponibili alla ricezione
di nuovi pacchetti. Da questa modifica derivano due principali benefici: il fenomeno di buffer overflow sopra descritto viene mitigato in quanto il nodo
che ha diversi pacchetti bufferizzati da processare sarà considerato solo dopo i nodi che hanno meno pacchetti nel loro buffer di ricezione, inoltre migliora
le performance della rete in quanto i nodi saranno considerati in base alla loro disponibilità ad accogliere nuovi pacchetti.\\

Nella versione base del protocollo ciascun nodo risponde alla ricezione di un messaggio di wake up a lui destinato attivando la radio principale e mettendosi
in attesa di ricevere nuovi pacchetti. Nella versione qui proposta si introduce la possibilità da parte del nodo di prendere la decisione se svegliarsi o meno.
In particolare, si utilizzano \emph{energy predictor} per studiare le predizioni dell'energia derivata da sorgenti esterne al fine di determinare se un nodo sarà o
meno nelle condizioni di potersi svegliare. La decisione viene presa dal nodo in questione in maniera del tutto indipendente dal resto della rete e basa interamente
la sua decisione in base a quanta energia sarà in grado di derivare da sorgenti esterne quali pannelli solari o turbine eoliche. Nel dettaglio, al momento
della ricezione della sequenza di wake up il nodo sceglie con una probabilità $p$, che è direttamente proporzionale al valore di energia predetto,
se questo dovrà attivare la radio principale oppure rimanere inattivo (SLEEP). L'idea in questo caso è che una sequenza di wake up può interessare più nodi
in ricezione e dal momento che alla fine solamente un nodo sarà selezionato come intermediario allora questa decisione può essere influenzata dalla
disponibilità energetica dei nodi stessi in un certo istante \emph{T}. Si noti come nel caso migliore la scelta del nodo intermediario coinvolgerà un numero
inferiore di nodi rispetto alla versione originale, con conseguente diminuizione dei costi energetici e latenze dovute a possibili ritrasmissioni.
Può tuttavia accadere che il nodo candidato alla ricezione non si svegli mai in quanto il valore randomico $r$ non supera mai la soglia $t$ da noi fissata.
Ma questo accade con una probabilità che dipende dal numero di possibili ritrasmissioni $n$ per livello e che diminuisce esponenzialmente ad ogni tentativo.
Nello scenario considerato, con $n=3$ è possibile ottenere un \emph{Packet Delivery Ratio} (PDR) medio del 99-100\% che permette parallelamente di abbattere i
consumi energetici dal 7\% al 13\%.\\

\begin{figure}
    \begin{center}
        \includegraphics[scale=0.5]{energy_plot.png}
        \caption{Benchmark comparativo dei consumi energetici dei due protocolli}
    \end{center}
\end{figure}

I risultati a cui si fa riferimento all'interno di questo documento fanno tutti riferimento ad uno scenario in cui si utilizza un terreno di dimensioni
$100 \times 100$, il sink node è piazzato nel centro, sono presenti 140 nodi uniformemente distribuiti all'interno dello scenario e varia il valore
dell'\emph{inter-arrival time} (\emph{iaTime}) per regolare la frequenza di generazione ed invio di dati da parte dei nodi della rete. Inoltre tutti
i nodi sono forniti di pannelli solari per derivare nuova energia dall'ambiente esterno e, nel caso della variante impiegata, gli energy predictor
vengono allenati per circa 3 giorni prima dell'inizio della simulazione per poter generare predizioni sui valori di energia predetti in futuro.\\

È chiaro che il principale vantaggio della variante qui proposta è l'efficienza energetica. Infatti, il suo principale obbiettivo è quello di migliorare i consumi
energetici di GREEN-WUP. Si noti come la variante proposta migliori l'efficienza energetica di GREEN-WUP senza peggiorarne le performance in generale,
mantendendo ad esempio il PDR e la latenza inalterati. Inoltre, in particolari casi di traffico intenso della rete (\emph{iaTime} $\sim$ 2-5) riesce anche a
ottenere risultati migliori rispetto a quest'ultimo.

\begin{figure}
    \centering
    \begin{minipage}{.5\textwidth}
      \centering
      \includegraphics[width=1\linewidth]{pdr_plot.png}
      \caption*{(a)}
    \end{minipage}%
    \begin{minipage}{.5\textwidth}
      \centering
      \includegraphics[width=1\linewidth]{latency_plot.png}
      \caption*{(b)}
    \end{minipage}
    \caption{Comparazione fra i due protocolli: (a) il PDR e (b) le latenze ottenute. }
    \end{figure}
\phantomsection % Give this command only if hyperref is loaded
\end{document}